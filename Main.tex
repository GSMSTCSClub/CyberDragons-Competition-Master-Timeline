% Options for packages loaded elsewhere
\PassOptionsToPackage{unicode}{hyperref}
\PassOptionsToPackage{hyphens}{url}
\PassOptionsToPackage{dvipsnames,svgnames,x11names}{xcolor}
%
\documentclass[
  letterpaper,
  DIV=11,
  numbers=noendperiod]{scrartcl}

\usepackage{amsmath,amssymb}
\usepackage{iftex}
\ifPDFTeX
  \usepackage[T1]{fontenc}
  \usepackage[utf8]{inputenc}
  \usepackage{textcomp} % provide euro and other symbols
\else % if luatex or xetex
  \usepackage{unicode-math}
  \defaultfontfeatures{Scale=MatchLowercase}
  \defaultfontfeatures[\rmfamily]{Ligatures=TeX,Scale=1}
\fi
\usepackage{lmodern}
\ifPDFTeX\else  
    % xetex/luatex font selection
  \setmainfont[]{Inter}
  \setsansfont[]{Inter}
\fi
% Use upquote if available, for straight quotes in verbatim environments
\IfFileExists{upquote.sty}{\usepackage{upquote}}{}
\IfFileExists{microtype.sty}{% use microtype if available
  \usepackage[]{microtype}
  \UseMicrotypeSet[protrusion]{basicmath} % disable protrusion for tt fonts
}{}
\makeatletter
\@ifundefined{KOMAClassName}{% if non-KOMA class
  \IfFileExists{parskip.sty}{%
    \usepackage{parskip}
  }{% else
    \setlength{\parindent}{0pt}
    \setlength{\parskip}{6pt plus 2pt minus 1pt}}
}{% if KOMA class
  \KOMAoptions{parskip=half}}
\makeatother
\usepackage{xcolor}
\setlength{\emergencystretch}{3em} % prevent overfull lines
\setcounter{secnumdepth}{5}
% Make \paragraph and \subparagraph free-standing
\ifx\paragraph\undefined\else
  \let\oldparagraph\paragraph
  \renewcommand{\paragraph}[1]{\oldparagraph{#1}\mbox{}}
\fi
\ifx\subparagraph\undefined\else
  \let\oldsubparagraph\subparagraph
  \renewcommand{\subparagraph}[1]{\oldsubparagraph{#1}\mbox{}}
\fi


\providecommand{\tightlist}{%
  \setlength{\itemsep}{0pt}\setlength{\parskip}{0pt}}\usepackage{longtable,booktabs,array}
\usepackage{calc} % for calculating minipage widths
% Correct order of tables after \paragraph or \subparagraph
\usepackage{etoolbox}
\makeatletter
\patchcmd\longtable{\par}{\if@noskipsec\mbox{}\fi\par}{}{}
\makeatother
% Allow footnotes in longtable head/foot
\IfFileExists{footnotehyper.sty}{\usepackage{footnotehyper}}{\usepackage{footnote}}
\makesavenoteenv{longtable}
\usepackage{graphicx}
\makeatletter
\def\maxwidth{\ifdim\Gin@nat@width>\linewidth\linewidth\else\Gin@nat@width\fi}
\def\maxheight{\ifdim\Gin@nat@height>\textheight\textheight\else\Gin@nat@height\fi}
\makeatother
% Scale images if necessary, so that they will not overflow the page
% margins by default, and it is still possible to overwrite the defaults
% using explicit options in \includegraphics[width, height, ...]{}
\setkeys{Gin}{width=\maxwidth,height=\maxheight,keepaspectratio}
% Set default figure placement to htbp
\makeatletter
\def\fps@figure{htbp}
\makeatother

\usepackage{tikz}
\newfontfamily\sectionfont[Color=000000]{Inter}
\newfontfamily\subsectionfont[Color=000000]{Inter}
\newfontfamily\subsubsectionfont[Color=000000]{Inter}
\addtokomafont{section}{\sectionfont}
\addtokomafont{subsection}{\subsectionfont}
\addtokomafont{subsubsection}{\subsubsectionfont}
\usepackage{fvextra}
\usepackage{svg}
\DefineVerbatimEnvironment{Highlighting}{Verbatim}{breaklines,commandchars=\\\{\}}
\KOMAoption{captions}{tableheading}
\makeatletter
\makeatother
\makeatletter
\makeatother
\makeatletter
\@ifpackageloaded{caption}{}{\usepackage{caption}}
\AtBeginDocument{%
\ifdefined\contentsname
  \renewcommand*\contentsname{Table of contents}
\else
  \newcommand\contentsname{Table of contents}
\fi
\ifdefined\listfigurename
  \renewcommand*\listfigurename{List of Figures}
\else
  \newcommand\listfigurename{List of Figures}
\fi
\ifdefined\listtablename
  \renewcommand*\listtablename{List of Tables}
\else
  \newcommand\listtablename{List of Tables}
\fi
\ifdefined\figurename
  \renewcommand*\figurename{Figure}
\else
  \newcommand\figurename{Figure}
\fi
\ifdefined\tablename
  \renewcommand*\tablename{Table}
\else
  \newcommand\tablename{Table}
\fi
}
\@ifpackageloaded{float}{}{\usepackage{float}}
\floatstyle{ruled}
\@ifundefined{c@chapter}{\newfloat{codelisting}{h}{lop}}{\newfloat{codelisting}{h}{lop}[chapter]}
\floatname{codelisting}{Listing}
\newcommand*\listoflistings{\listof{codelisting}{List of Listings}}
\makeatother
\makeatletter
\@ifpackageloaded{caption}{}{\usepackage{caption}}
\@ifpackageloaded{subcaption}{}{\usepackage{subcaption}}
\makeatother
\makeatletter
\@ifpackageloaded{tcolorbox}{}{\usepackage[skins,breakable]{tcolorbox}}
\makeatother
\makeatletter
\@ifundefined{shadecolor}{\definecolor{shadecolor}{rgb}{.97, .97, .97}}
\makeatother
\makeatletter
\makeatother
\makeatletter
\makeatother
\ifLuaTeX
  \usepackage{selnolig}  % disable illegal ligatures
\fi
\IfFileExists{bookmark.sty}{\usepackage{bookmark}}{\usepackage{hyperref}}
\IfFileExists{xurl.sty}{\usepackage{xurl}}{} % add URL line breaks if available
\urlstyle{same} % disable monospaced font for URLs
\hypersetup{
  pdftitle={2023-2024 GSMST CS Club CyberDragons Master Timeline},
  pdfauthor={Anish Goyal (Chief of Competitions); Bibek Bhattari (CyberDragons Head); Andrew Zeng (CyberDragons Assistant Head)},
  colorlinks=true,
  linkcolor={blue},
  filecolor={Maroon},
  citecolor={Blue},
  urlcolor={Blue},
  pdfcreator={LaTeX via pandoc}}

\title{2023-2024 GSMST CS Club CyberDragons Master Timeline}
\author{Anish Goyal (Chief of Competitions) \and Bibek Bhattari
(CyberDragons Head) \and Andrew Zeng (CyberDragons Assistant Head)}
\date{}

\begin{document}
\maketitle
\begin{tikzpicture}[overlay,remember picture]
  \node[anchor=center, opacity=0.3] at ([yshift=-2cm]current page.center) {\includegraphics[width=2\paperwidth]{CyberDragons.png}};
\end{tikzpicture}
\ifdefined\Shaded\renewenvironment{Shaded}{\begin{tcolorbox}[frame hidden, sharp corners, breakable, boxrule=0pt, interior hidden, enhanced, borderline west={3pt}{0pt}{shadecolor}]}{\end{tcolorbox}}\fi

\renewcommand*\contentsname{Table of Contents}
{
\hypersetup{linkcolor=}
\setcounter{tocdepth}{6}
\tableofcontents
}
\tableofcontents
\begin{tikzpicture}[overlay,remember picture]
  \node[anchor=center, opacity=0.3] at ([yshift=-2cm]current page.center) {\includegraphics[width=2\paperwidth]{CyberDragons.png}};
\end{tikzpicture}
\newpage{}

\hypertarget{overview}{%
\section{Overview}\label{overview}}

This document provides a comprehensive overview of all of the deadlines
and events for the 2023-2024 GSMST CS Club CyberDragons. It is intended
to be used as an ``at-a-glance'' reference for mentors and competitors,
as all of these same deadlines and events can be found on both the
\href{https://drive.google.com/file/d/1m0VGlUpv1V3IJLBle5qOZae0BnPgdrRq/view?usp=sharing}{\textbf{proposal
document}} and \textbf{club calendar}. Keep in mind that not all the
dates have been finalized yet, and it is important to periodically check
the competition sites for any date changes. Also, if there aren't any
prizes written in the description, then that just means that the prizes
haven't been finalized yet or there aren't any cash prizes. This
document will be ordered by the competitions

\hypertarget{cyberpatriot}{%
\section{CyberPatriot}\label{cyberpatriot}}

\hypertarget{overall-logistics}{%
\subsection{Overall Logistics}\label{overall-logistics}}

\textbf{Competition Period:} October 21---January 20

\textbf{Registration Period:}\\
- Team Registration: May 31---October 5\\
- Rosters Finalized: November 1\\
- Payment Deadline: November 15

\textbf{Fees:} \$205 per team, all girls teams are free

\textbf{Limit:} 5 teams, 6 members per team

\href{https://www.uscyberpatriot.org/}{CyberPatriot} is such a big thing
for CyberDragons that we have multiple deadlines to prepare for the
competitions, as well as the actual \emph{dates} of the competitions
themselves.

\hypertarget{qualifying-images-september-1september-18}{%
\subsection{Qualifying Images (September 1---September
18)}\label{qualifying-images-september-1september-18}}

Every year, we release a set of ``qualifying images'' that applicants
are required to do one of to be in CyberPatriot (along with the short
answer Google Form and paying \$30). The release date for these images
will be \emph{during our first CyberDragons meeting} on September 1.
They will have until September 18 at 8 AM to score as many points as
they can, or about 2.5 weeks. But you may be wondering: what's the
incentive for our applicants to spend more than the required six hours
on the qualifying image of their choice other than the fact that it's
required and it will ``make them better?'' Well, we are implementing a
mini-competition system here. Whoever gets the most points out of all of
the qualifying images, they will receive a special prize. And the top 6
performing members overall will join Team Zeta, the fifth team dedicated
to do as well as possible in the competition. The entire competition
will be live. There will be a scoring server that remotely scores each
image, containing the applicants' unique identifiers (aka their first
name and last initial), which will deactivate at precisely 8 AM on
September 18. This means that each participant's progress will be
tracked \emph{live}, through a Discord webhook, and everyone will be
able to monitor the progress of themselves relative to their peers
during the competition. Everyone's progress will be known to everyone
else, and there will be no need to have them submit screenshots of their
scores to the GSMST CS Club email anymore.

\hypertarget{official-practice-round-october-13}{%
\subsection{Official Practice Round (October
13)}\label{official-practice-round-october-13}}

We will meet after school on October 13 to compete in CyberPatriot's
official CyberPatriot practice round! And since the image difficulties
for this practice round will be 100x easier than the qualifying images,
this will definitely serve as a morale boost for all the CyberDragons
members, since they now feel confident that they're actually performing
\emph{well}. But the practice round is also important for another
reason: it serves as the only exposure CyberDragons will have to the
\emph{Boeing CyberPhysical Systems Challenge} before the semifinals
round (assuming everyone places in platinum). Therefore, it's important
to figure out what the Boeing challenge really is, and how we plan to
attack it in the future.

\hypertarget{round-1-october-21}{%
\subsection{Round 1 (October 21)}\label{round-1-october-21}}

The first round of CyberPatriot will take place on Saturday, October 21
from 9 AM---2 PM. Considering that the competition is now only
\emph{four} hours long instead of six, we shortened the time by an hour.
This means that teams will have a little bit extra time at the end to
improve their writeups and write reflections. Whether we make students
pay for pizza or whether that is covered by the \$30 cost is up to you,
but I say \$5 for pizza per person just like last year is a good plan.

\hypertarget{round-2-november-4}{%
\subsection{Round 2 (November 4)}\label{round-2-november-4}}

The second round of CyberPatriot will take place on Saturday, November 4
from 9 AM---2 PM.

\hypertarget{state-round-december-2}{%
\subsection{State Round (December 2)}\label{state-round-december-2}}

The state round of CyberPatriot will take place on Saturday, December 2
from 9 AM---2 PM.

\hypertarget{cyberpatriot-semifinals-january-20}{%
\subsection{CyberPatriot Semifinals (January
20)}\label{cyberpatriot-semifinals-january-20}}

CyberPatriot semifinals (for qualifying teams) will take place on
Saturday, January 20 from 9 AM---2 PM.

\hypertarget{csaw-ctf}{%
\section{CSAW CTF}\label{csaw-ctf}}

\hypertarget{logistics}{%
\subsection{Logistics}\label{logistics}}

\textbf{Competition Period:} September 8---10

\textbf{Registration Period}: August---September 10

\textbf{Fees:} None

\textbf{Limit:} None

\hypertarget{description}{%
\subsection{Description}\label{description}}

\href{https://www.csaw.io/ctf}{CSAW CTF} is one of the oldest and
biggest jeopardy-style CTFs. This competition is for students who are
trying to go into the field of security, as well as for advanced
students and industry professionals who want to practice their skills.
The qualifiers take place from September 8 to September 10, but high
schoolers cannot go past the qualifier round.

\hypertarget{asis-ctf}{%
\section{ASIS CTF}\label{asis-ctf}}

\hypertarget{logistics-1}{%
\subsection{Logistics}\label{logistics-1}}

\textbf{Competition Period}: September 23---24 \& December 30---31

\textbf{Registration Period}: August---September 24

\textbf{Fees:} None

\textbf{Limit:} None

\hypertarget{description-1}{%
\subsection{Description}\label{description-1}}

\href{https://asisctf.com/}{ASIS CTF} is an annual jeopardy-style CTF
competition organized by the ASIS (Academy or Skills and Information
Security) team. The problems include: general security infomration
(trivia), web hacking, modern cryptography, exploit, forensics, reverse
engineering, steganography, etc. The qualification round is in late
September and finals are in December.

\hypertarget{cyberstart-america}{%
\section{CyberStart America}\label{cyberstart-america}}

\hypertarget{logistics-2}{%
\subsection{Logistics}\label{logistics-2}}

\textbf{Competition Period}: October---April

\textbf{Registration Period:} October---April

\textbf{Fees:} None

\textbf{Limit:} None

\hypertarget{description-2}{%
\subsection{Description}\label{description-2}}

A cybersecurity learning platform (CyberStart) based in the United
Kingdom has expanded their reach to the United States, but since
Americans won't be motivated without a little money incentive, they are
offering cash prize for winners! This competition is asynchronous and
lasts from October---April, which is basically the entire year. And
technically, CyberDragons isn't even focusing on this competition: it's
\href{https://docs.google.com/document/d/17gccH6BFHdis-tRGBOMymy97Gh0RO_vw79blIst_hi4/edit?usp=sharing}{Beginner's
Programming} that is. We're just telling CyberDragons members to
participate because it would be good introductory material for beginners
and a piece of cake for our more experienced members.

\hypertarget{dicectf}{%
\section{DiceCTF}\label{dicectf}}

\hypertarget{logistics-3}{%
\subsection{Logistics}\label{logistics-3}}

\textbf{Competition Period}: February 6---7

\textbf{Registration Period}: February 1---6

\textbf{Fees:} None

\textbf{Limit:} None

\hypertarget{description-3}{%
\subsection{Description}\label{description-3}}

\href{https://ctf.dicega.ng/}{DiceCTF} is an annual jeopardy-style CTF
competition hosted every year by \emph{DiceGang}, a prestigious
competitive hacking group. The prize pool is \$5000, and it is open to
all students.

\hypertarget{la-ctf}{%
\section{LA CTF}\label{la-ctf}}

\hypertarget{logistics-4}{%
\subsection{Logistics}\label{logistics-4}}

\textbf{Competition Period:} February 10---February 12

\textbf{Registration Period}: TBD

\textbf{Fees}: None

\textbf{Limit}: None

\hypertarget{description-4}{%
\subsection{Description}\label{description-4}}

\href{https://lactf.uclaacm.com/}{LA CTF} is a jeopardy-style CTF
competition hosted by UCLA. There is no size limit on teams, and
non-UCLA students must compete in the \emph{Open} division. The
competition consists of a variety of competitive cybersecurity
challenges in addition to relaxed events like typing competitions. It
takes place in mid-February for 42 hours.

What we think is really cool about this CTF competition specifically is
that there are also prizes awarded for the best write ups to the
challenges of the competition, and write ups include video walkthroughs.
So, if we really wanted to tryhard this CTF, we could just make a bunch
of YouTube videos with really high production value and potentially get
some prizes from doing that alone. But the prizes for winning the actual
competition are \$500, \$300, \$200, and \$100 for 1st, 2nd, 3rd, and
4/5th place, respectively.

Another really cool thing about this competition is that there are going
to be speakers on the UCLA campus streamed live for all the virtual
participants. A notable name is John Hammond, who has his own YouTube
channel for cybersecurity stuff.

\hypertarget{us-cyber-challenge-cyber-quests}{%
\section{US Cyber Challenge: Cyber
Quests}\label{us-cyber-challenge-cyber-quests}}

\hypertarget{logistics-5}{%
\subsection{Logistics}\label{logistics-5}}

\textbf{Competition Period}: February 12---March 26

\textbf{Registration Period}: January 29---March 26

\textbf{Fees:} None

\textbf{Limit:} None

\hypertarget{description-5}{%
\subsection{Description}\label{description-5}}

\href{https://uscc.cyberquests.org/}{Cyber Quests} (not to be confused
with LM's \emph{CyberQuest}) is a online competition allowing
participants to demonstrate their knowledge in a variety of information
security realms. The main benefit of this competition is that it has a
niche quiz (the entire competition is basically a quiz) that deals a lot
with networking. There are no ``teams'' for this either; you have to
create an account and participate individually and asynchronously,
similar to CyberStart America.

\hypertarget{lockheed-martins-cyberquest}{%
\section{Lockheed Martin's
CyberQuest}\label{lockheed-martins-cyberquest}}

\hypertarget{logistics-6}{%
\subsection{Logistics}\label{logistics-6}}

\textbf{Competition Period}: March 4

\textbf{Registration Period}: September/October (estimated)---February 3

\textbf{Fees:} None

\textbf{Limit:} 2 team limit for virtual and in-person each (4 max), up
to three members per team

\hypertarget{description-6}{%
\subsection{Description}\label{description-6}}

\href{https://www.lockheedmartin.com/en-us/who-we-are/communities/cyber-quest.html}{CyberQuest}
is a cloud-based competition created by Lockheed Martin cybersecurity
engineers. There will be multi-step intrustion, steganography, reverse
engineering, full OS hack, packet capture, and web exploitation. They
will provide users with desktops that have Kali \& Windows installed,
both in-person and online.

\hypertarget{technology-student-associations-cybersecurity-event}{%
\section{Technology Student Association's Cybersecurity
Event}\label{technology-student-associations-cybersecurity-event}}

\hypertarget{logistics-7}{%
\subsection{Logistics}\label{logistics-7}}

\textbf{Competition Period}: March 13---16

\textbf{Registration Period}: August---February 16

\textbf{Fees:} \$250 per student + food

\textbf{Limit:} 5 people

\hypertarget{description-7}{%
\subsection{Description}\label{description-7}}

\href{https://docs.google.com/document/d/1q5fqitr-ezFVO6owV2g9lBzOqPFntbkny2nBKQRG9OU/edit?usp=sharing}{Technology
Student Association (TSA)'s Cybersecurity event} is an asynchronous
jeopardy-style CTF challenge that is essentially picoCTF's little
cousin. The idea here is the following: if we can get as many
CyberDragons members into the Cybersecurity event as possible (since
there's technically no limit on how many people can participate in an
event, just who can go to SLC), we can get first place at SLC. This
would, of course, require commitment to \emph{both} the GSMST
CyberDragons and TSA, and the participants would have to pay for
themselves, but if they are fine with that, it \emph{should} be fine.
And TSA will probably take care of registration, so no extra paper work!

\hypertarget{picoctf}{%
\section{picoCTF}\label{picoctf}}

\hypertarget{logistics-8}{%
\subsection{Logistics}\label{logistics-8}}

\textbf{Competition Period:} March 14---March 28

\textbf{Registration Period}: February 1---March 28

\textbf{Fees}: None

\textbf{Limit}: 5 students max per team, no limit on teams

\hypertarget{description-8}{%
\subsection{Description}\label{description-8}}

\href{https://picoctf.org/}{picoCTF} is the biggest CTF competition in
the entire world. It's open to people of varying skill levels and known
for its high prize pool. It's hosted by Carnegie Mellon, and placing
within the top three teams gets you an all-expenses-paid campus tour of
CMU. Last year, we got eighth place in the High School Divison and we
plan to blow everyone's expectations away this year.

\hypertarget{cyber-skylines-national-cyber-league-ncl}{%
\section{Cyber Skyline's National Cyber League
(NCL)}\label{cyber-skylines-national-cyber-league-ncl}}

\hypertarget{logistics-9}{%
\subsection{Logistics}\label{logistics-9}}

\textbf{Competition Period:} March 31---April 2 \& April 14---April 16

\textbf{Registration Period:} January---April 13

\textbf{Fees:} \$35/team

\textbf{Limit:} Max of 7 students/team

\hypertarget{description-9}{%
\subsection{Description}\label{description-9}}

\href{https://cyberskyline.com/events/ncl}{Cyber Skyline's National
Cyber League (NCL)} is a virtual cybersecurity competition and training
platform. It is designed to provide participants, including students and
professionals, with hands-on experience and practical skills in various
cybersecurity disciplines. The NCL features a series of challenges that
simulate real-world scenarios, allowing participants to develop and
demonstrate their expertise in areas such as network security,
cryptography, web application security, log analysis, and more. The NCL
follows a gamified approach, where participants compete individually or
in teams to solve a series of challenges within a given timeframe. These
challenges are designed to assess participants' technical knowledge,
critical thinking, problem-solving abilities, and ability to work under
pressure. The platform offers different difficulty levels, ranging from
introductory to advanced, catering to participants with varying skills.

There are two competitions offered in the NCL event: the individual game
and the team game. The individual game takes place from March 31 to
April 2, and you can register whenever you want for the individual game,
even during the competition period itself. For the team game, the
competition period is April 14 to April 16, and you have to register
between January to April 13 by 11:59 PM. It costs \$35 per team for a
max of two teams. We could recruit two teams and make each team member
pay a hefty little fine along with that. And there's 7 members max per
team. This competition also focuses on log analysis, network traffic
analysis, etc.

\hypertarget{uxe5ngstromctf}{%
\section{ÅngstromCTF}\label{uxe5ngstromctf}}

\hypertarget{logistics-10}{%
\subsection{Logistics}\label{logistics-10}}

\textbf{Competition Period:} April 21---26

\textbf{Registration Period}: TBD

\textbf{Fees:} None

\textbf{Limit:} None

\hypertarget{description-10}{%
\subsection{Description}\label{description-10}}

\href{https://angstromctf.com/}{ÅngstromCTF} is another jeopardy-style
CTF competition hosted by team Ångstrom. It holds a focus on binary
exploitation, cryptography, reverse engineering, and web exploitation,
with a few miscellanous questions.

\hypertarget{tjctf}{%
\section{TJCTF}\label{tjctf}}

\hypertarget{logistics-11}{%
\subsection{Logistics}\label{logistics-11}}

\textbf{Competition Period:} May 26---28

\textbf{Registration Period}: TBD

\textbf{Fees}: None

\textbf{Limit}: None

\hypertarget{description-11}{%
\subsection{Description}\label{description-11}}

\href{https://tjctf.org/}{TJCTF} is a jeopardy-style CTF competition
that takes place in late May (probably after school is out/during
finals). The categories in this competition are: cryptography, binary
exploitation, reverse engineering, web exploitation, forensics, etc.
Even though it's seemingly \emph{impossible} to compete in this
challenge due to its timeframe being so close to the end of the school
year, I definitely think it is worth a try. The prize pool is \$4000,
with \$1,500, \$1,000, \$750, \$500, and \$200, being awarded to the
winners in order, respectively.



\end{document}
